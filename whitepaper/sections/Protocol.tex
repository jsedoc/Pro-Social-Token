\section{\textbf{The Protocol}}

\subsection{Introduction}

\subsection{Validation Mining: Proof of Human Work}


\subsection{Social Mining: Proof of Prosocial Work}

\subsection{Formal Statement of Problem}


Rewarding actions that impact communities positively is difficult for at least two fundamental reasons. First, it is difficult to define what positive means. Free facial tattoos can be great or terrible for society depending on who you ask. Secondly, such actions are difficult to quantify. What is the price on a heart-warming on-line comment that helps someone be a better person? It would also be nice if such actions could be awarded sooner rather later. This isn't to say that we should monetize all positive activity but it certainly seems useful that good folks are rewarded in some way.

Virtual communities make this problem much more feasible. People with similar tastes congregate in groups where they affirm each other's actions through retweets, upvotes, likes and other similar votes of confidence (henceforth refered to as upvotes). In addition, these groups judge each other as well. Presumably, a community of vegans would value upvotes earned in a vegetarian community versus one earned in a community of steak lovers. 

ProSocial attempts to promote good social behavior through an ecosystem which rewards affirmation in an online community with a hope that this expands or at least, spills over into more broadly defined communities. We use Reddit's subreddits as an example but the vision is to compare Reddit to Facebook and eventually, to real communities.

\section*{Buying In}

Companies and individuals can demonstrate their contributions to the cause by simply burning (instead of holding) ProSocial tokens. This is effectively a no-refund donation to ProSocial. We call ProSocial tokens, TOKENs while we think of a better name. 

\section*{Protocol}

There are three ways to obtains TOKENs: Trading, verification mining, and social mining. 

\section*{Trading}

Standard protocol depending on which platform we use. Not much more to add beyond that.

\section*{Blockchain Mining: Token Verification}

Verification mining confirms transactions between TOKEN holders and confirm that a prosocial event took place. Miners earn a transaction fee but there is no generation of new TOKENS from mining. The only way to generate new TOKENS is by Social Mining and this is literally performing acts that are valued by others. Either way, the ledger contains three fundamental types of information.

\begin{enumerate}
	\item Transactions between addresses (Andy sent 5 TOKENs to Bob). This includes rewards from social mining. This bit is probably copy-paste from existing work. We do not innovate here.
	
	\item ProSocial Actions (eg. Andy upvoted Bob). Any participating platform must adhere to a set of protocol rules which we will spell out later. One of these must be some sort of an API to report actions to verification miners. The reason why we need verification miners is that we cannot trust a single entity to report on which actions TOOK place (and not which actions are socially valuable). This bit is again quite standard. We do not innovate here either although this is an essential part of the system.
	
	\item Preference orderings of which communities are most valued and who performs most valuable actions within communities. These orderings provide a basis for rewards from social mining as well as an exchange rate for devaluation or transfer of tokens between communities. We innovate here. Social mining performing tasks that are valued by communities. Social mining also involves voting on which communities are doing better work. 
	
\end{enumerate}

\textbf{KEY:} The difficulty for verification mining is adjusted so that aggregate gain from social mining is greater than the gain from verification mining by some factoer that is either a constant or perhaps, determined by society. Who knows, maybe all we care about is verification mining. Either way, the point is to allow social mining.

\section*{Social Mining: One Dimension Preferences}

\begin{enumerate}
	\item Assume for now that individuals have preferences in a single dimension and hence, those who do the most to contribute get upvoted more. More concretely, more upvotes in Reddit shows that someone has contributed a lot and someone in r/ihatepuppies wouldn't care that the upvotes are gained from r/awww.
	
	\item Rewarding the accumulation of upvotes via TOKENs to an address automatically generates prosocial behavior. Someone who earns upvotes also earns TOKENs which are presumably worth something. More on this later.
	
	\item TOKEN distribution is probably straightforward in the reddit setting. Ownership of a reddit account can easily be anonymously verified by the reddit account posting a hashed message to some subreddit for example. No one actually knows how much TOKEN a username has because you simply cannot link accounts to the usernames publicly. Only the blockchain and the user knows. 
	
	\item \textbf{It might be possible to reverse engineer identities if token distributions are public and there are no trades. If there are sufficient trades, I think the SnR from convoluting transactions with rewards would obscure the identity of who is being rewarded (if this even matters). This or, we can additionally make the token rewarding noisy.} 
	
	\item \textbf{Platform adoption needs to feature a way of linking the user on the platform to the ProSocial ecosystem. ProSocial's job might be to create some APIs allowing any platform to monetize their member's voting behaviors.}
	
\end{enumerate}

So far, we have a situation where we assume everyone agrees on what is good and what is not. This is too simplistic. We need to judge each other with equally valid subjective preferences however nutty they are.

\section*{Social Mining: Many Dimension Preferences}

\begin{enumerate}
	\item 3 subreddits. One subreddit is r/bots and all they do is to repost and somehow, they love it. The second subreddit are r/ihatepuppies and the third subreddit is r/awww.
	
	\item Assume society is mostly in r/awww. Hence, r/awww would list their preferences are (awww,bots,ihatepuppies). ihatepuppies prefers (ihatepuppies, bots, awww) and bots prefer (bots, awww,ihatepuppies).
	
	\item Within each subreddit, people just do what they like and get rewarded with TOKENs in a given interval. \textbf{We need to think about how much TOKENs to reward and perhaps have a dynamic system of reward or exchange. I think it would be most interesting to have the market pin down all rewards.} For now, think of users in each subreddit accumulating TOKENs doing what each subreddit values.
	
	\item Another part of social mining is to reaffirm others. Each user is presented with quasi-random choice of TOKENs to take (and to give). Quasi-random because we need to understand preferences of the network. The choices need to be given to elicit those preferences. Clarification: To keep your tokens from depreciating, make a choice between Platforms X and Y every so often. Choosing X gives X some tokens and also gives the user some tokens for the effort. Over time, platforms that are universally liked grow and the ones that are hated shrink in rewards.
	
	\item Let the reward from revealing preferences accumulate in a decreasing returns fashion over time. Voting every second is best, but if you vote once every 10 seconds, your vote is worth 90 pct of the vote of the guy who votes once every 10 seconds. Congestion might be an issues. \textbf{Trade off is congestion vs eliciting complete preferences.}
	
	\item Suppose there are many users in r/awww and a few each in r/ihatepuppies and r/bots.
	
	\item Each time, assuming that economic incentives DO NOT affect voting choices, it is clear that r/awww gets richer more quickly. \textbf{We can then adjust the reward so that the inequality is capped. This is a classic growth vs inequality issue. There is no perfect answer here. Just pick something, or let the community decide.}
	
\end{enumerate}

At this point, assuming no hacks, people are rewarded for good behavior, and communities are rewarded by positive behavior. Voting and giving honest answers is also rewarded. There is no economic incentive biasing stuff yet. Let's deal with that now.

\section*{Infallibility}
\begin{enumerate}
	\item \textbf{Hack 1: r/bots realize that a scripted upvote fest within r/bots makes money.} This is unacceptable to everyone, both r/awww and r/ihatepuppies now rank r/bots last in their preferences, punishing them. This downvoting of r/bots as a community can impact the trickle of TOKENs they earn from performing valued tasks in r/bots. 
	
	We have to tweak the rewards so that such aggregate downvoting breaks the behavior in r/bots. If people are misinformed and vote to give r/bots TOKENS while gaining TOKENs themselves even though it is bad for them and society, that is just too bad. First amendment again. We can't judge them. We are not in the business of dealing with fake news either. We just have to focus on monetizing behavior and not judge behavior.
	
	\textbf{The key bit here obviously is to make sure that preferences are elicited so that the bots can be punished. Perhaps, tokens cannot be traded until it is clear what preferences of society are on a particular address just like how options granted to management or employees cannot be exercised immediately.}
	
	\item \textbf{Hack 2: Bot guys join r/awww and just upvote themselves there.} This brings down the stock of r/awww and presumably r/awww can remove these bots, or downvote them. IF not, bots rule, and as we discussed before, we are not in the business to telling others what to do. 
	
	\item \textbf{Hack 3: Bots replicate themselves on a platform or replicate platforms.} Build into the API some verification task that bots cannot perform and this verification task is something that the community can upgrade over time by verification mining consensus. If the bots take over so fast that they take over the Internet, we should just grab some popcorn, head to the beach and wait for the matrix to play itself out. Either way, ensuring that HUMAN behavior or at least, HUMAN tolerated bot behavior is being measured is something that we need to ensure because we cannot trust the platform itself to measure bots. The participating platform can produce the worse of humanity but voting can take care of that and our job is to facilitate voting, no more. If a platform produces bots and this is not prevented, then bots are indistinguishable from humans and this breaks the system. In short, we cannot trust evilreddit.com to have a backdoor in their human verification system.
	
\end{enumerate}    

\section*{Key innovation: Verification and Social Mining}

We need to think how to set the supply and demand so that economic rewards from prosocial behavior is equal to the economic rewards from mining. Meaning, the marginal profit from operating a TOKEN mining operation next to a free energy source is equal to giving out free hugs. That way, the economic surplus of mining equals the economic surplus of not being an asshole. A miner who makes profits is supported by good social behavior of equal magnitude. We don't want ProSocial to be just another token that benefits miners. We want social mining to be the main way TOKENs are created.

\section*{What are TOKEN for besides braggery?}
They are claims (perhaps with a lag like options as mentioned before) to something useful. Remember the proof of burn from earlier? You can burn tokens by buying them and burning them. You simply provably support the ProSocial ecosystem. 

An endowment from the burned assets can support a steady trickle of funds disbursed in a different store of value: (litecoin, ether, whatever). So more TOKENs in a ProSocial person's account gives the ProSocial person a trickle of interest the burned assets generate. This allows them to continue giving free hugs.

\newpage
\newpage
