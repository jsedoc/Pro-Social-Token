\section{\textbf{Introduction}}

\subsection{Overview}

Over the last thirty years, information technology has touched every aspect of society. Social media gave real identity to internet users, Google organized the world of information, and artificial intelligence promises to transform every aspect of society. And yet, some of the most pressing concerns of our society remains stubbornly untouched by technology:

\begin{enumerate}
	\item The federal government collects over three trillions dollars in tax dollar per year, much of this dollar is spent constructing and maintain “public goods.” And yet there is no transparency or accountability in how the funds are allocated
	\item Each year, over 300 billion dollars are donated to charities in America, and the end benefactors of the funds are similarly opaque. 
	\item People are inherently selfish and will act against the interest of the public good if not properly incentivized
	\item Prosocial people are not appropriately rewarded by the capitalist system as it is currently set up. For example teachers are grossly underpaid relative to CEOs.
	\item There is no system that tracks people’s reputation in a way that cannot be gamed, and no socially accepted justification to use such a system even if it exits.
\end{enumerate}


At Foundation, we believe all of the above problems are deeply inter-connected, and will solve them in one unified framework. We call it The Foundation Forum. The Forum is a marketplace with four components:

\begin{enumerate}
	\item A market with carefully crafted incentives so that prosocial behavior promoting the common good is properly valued and promoted.
	\item A decentralized ledger recording each participants preferences and transactions with the market, so that no central entity can control a person’s reputation, or corrupt an individual’s preferences. 
	\item Appropriate differential privacy measures so that individual’s identity is protected while their transactions are revealed.
	\item A liquidity valve to a secondary market where users may cash out their earnings from prosocial activities, so that prosocial behavior is appropriately rewarded financially. 

\end{enumerate}


% 
% Foundation Forum
% 
% One possible angle is that we are unbundling the actions of the government, the place we start is solving the common problem
% and we do so by constructing a marketplace that values altruistic actions, and we issue currency to reward said behaviors. 
% We record the transaction on a decentralized ledger so that no single party can manipulate the market of altruism for private gain.
% A natural consequence of this marketplace is that there is now a transparent and market-based way of measuring trust.
% Thus we have created an environment where people are known to act in good faith because they are rewarded for it by the market, 
% in essence we have a safe space that facilitates new channels of commerce.
% 
% Hit on the angle of opening a market for public goods
% 
% question: what is the annual expendidture on public goods? is it bigger than private goods?
% 
% 
% 

% \subsection{Problem Definition}

% Those who solve the "common problems" are not properly compensated

% \subsection{Protocol Overview}

% Key challenges: proof of human work, valuing the transaction, and exchange rate among different communities. We need to have an arugment of a permissionless blockchain in this vision. 

% \subsection{Value Proposition}

% 1. Properly Value work that is currently undervalued. 2. Properly measure and store trust. 3. Create an environment of good faith while not compromising privacy, thus facilitating commerce. 

\subsection{Paper Organization}

This document is organized as follows.

% In here we need to discuss the problem: 
% 	- define what prosocial means: giving more than taking
% 	- define how people are incentivize to care only for themselves - the common good problem
% 	- define how to determine if someone is trust worthy = prosocial
%   - define the storage of trust
%  	- Proposed protocol in broad strokes
% 	- potential application and value proposition
% 	- 
% 
% 
% Trust / [Prosocial]  / benevolence / helping: Zack preferes something more down-to-earth. Anneke: [Helping] is the best word that’s down-to-earth. Zack: [Compassion] is that a word? Anneke: it’s a sentiment, not an action. What about good-deeds. Zack: we are measuring the deed, not the person. We should note this, we are measuring the action, not the person. Anneke: yea it’s less contentious this way. Xiao: I like prosocial. But should we talk about measuring action vs person. Anneke: it’s a big part of branding. We are going for action, not the person. Artist might be just as a good of person as the person who’s moving, but it’s about the purpose not some intrinsict aspects. The mere act of creating art is not a prosocial act. Xiao:let’s expound on this, and build a brand around this. Anneke: We focus on the action. If we try to validate a person as a good person, then it’s hard.  People could accruve revenue really fast, it’s harder w/ action than reputational. Anneke: our society is set up for selfish creatures, there is not an inherent validation of prosocial actions, prosocial people make the least money: ie teachers, firemen. There is a subversivie element, but not hurting anyone. We are [presenting a way where people can be rewarded in a different way, money is not really a way].  This is another incentive other than warm glow. It can encourage altruist and non-altruist. What about [Counter-Currency]. Burning man would be a good place. 


\newpage
\newpage