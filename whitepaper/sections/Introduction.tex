\section{\textbf{Introduction}}


In 2017, the U.S. government collected 3.46 trillion dollars in taxes from the American Public, and it is well known that a significant portion of this revenue will be usurped by waste and fraud. Foundation will bring transparency and accountability to how public funds are allocated by tracking spending on a public decentralized ledger, and overtime unbundle the government of the United States of America.

We begin by designing a market to solve 







% One possible angle is that we are unbundling the actions of the government, the place we start is solving the common problem
% and we do so by constructing a marketplace that values altruistic actions, and we issue currency to reward said behaviors. 
% We record the transaction on a decentralized ledger so that no single party can manipulate the market of altruism for private gain.
% A natural consequence of this marketplace is that there is now a transparent and market-based way of measuring trust.
% Thus we have created an environment where people are known to act in good faith because they are rewarded for it by the market, 
% in essence we have a safe space that facilitates new channels of commerce.

\subsection{Problem Definition}

Those who solve the "common problems" are not properly compensated

\subsection{Protocol Overview}

Key challenges: proof of human work, valuing the transaction, and exchange rate among different communities. We need to have an arugment of a permissionless blockchain in this vision. 

\subsection{Value Proposition}

1. Properly Value work that is currently undervalued. 2. Properly measure and store trust. 3. Create an environment of good faith while not compromising privacy, thus facilitating commerce. 

\subsection{Paper Organization}

This document is organized as follows.

% In here we need to discuss the problem: 
% 	- define what prosocial means: giving more than taking
% 	- define how people are incentivize to care only for themselves - the common good problem
% 	- define how to determine if someone is trust worthy = prosocial
%   - define the storage of trust
%  	- Proposed protocol in broad strokes
% 	- potential application and value proposition
% 	- 
% 
% 
% Trust / [Prosocial]  / benevolence / helping: Zack preferes something more down-to-earth. Anneke: [Helping] is the best word that’s down-to-earth. Zack: [Compassion] is that a word? Anneke: it’s a sentiment, not an action. What about good-deeds. Zack: we are measuring the deed, not the person. We should note this, we are measuring the action, not the person. Anneke: yea it’s less contentious this way. Xiao: I like prosocial. But should we talk about measuring action vs person. Anneke: it’s a big part of branding. We are going for action, not the person. Artist might be just as a good of person as the person who’s moving, but it’s about the purpose not some intrinsict aspects. The mere act of creating art is not a prosocial act. Xiao:let’s expound on this, and build a brand around this. Anneke: We focus on the action. If we try to validate a person as a good person, then it’s hard.  People could accruve revenue really fast, it’s harder w/ action than reputational. Anneke: our society is set up for selfish creatures, there is not an inherent validation of prosocial actions, prosocial people make the least money: ie teachers, firemen. There is a subversivie element, but not hurting anyone. We are [presenting a way where people can be rewarded in a different way, money is not really a way].  This is another incentive other than warm glow. It can encourage altruist and non-altruist. What about [Counter-Currency]. Burning man would be a good place. 


\newpage
\newpage